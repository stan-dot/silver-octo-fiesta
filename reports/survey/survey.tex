
\documentclass{article}

\title{Dissertation Project Description}
\author{Stanislaw Malinowski}
\subitem{Rob Gaizauskas}
\subitem {COM3610}
\date{October 2022}

\begin{document}

\maketitle
\section{title}
Title, name, supervisor, module code, date,
and the following statement: This report is submitted in partial fulfilment of the requirement for the degree of [Degree Title] by [Full Name].

\section*{Contents}
\tableofcontents
\newpage

\section{Dissertation Project: Survey and Analysis Stage}

For students taking COM3610 the report should be between 3000 and 6000 words.


\section{Declaration}

All sentences or passages quoted in this report from other people's work have been specifically acknowledged by clear cross-referencing to author, work and page(s). Any illustrations that are not the work of the author of this report have been used with the explicit permission of the originator and are specifically acknowledged. I understand that failure to do this amounts to plagiarism and will be considered grounds for failure in this project and the degree examination as a whole

\newpage


\section{Abstract}

This should be two or three short paragraphs (100-150 words total), summarising the report.
A suggested flow is background, project aims, and achievements to date. It should not simply be a restatement of the original project outline.

\section{Chapter 1: Introduction}
- set the scene for the project by giving a little relevant background information - try to grab the reader's interest early.
-  Another is to clearly elucidate the aims and objectives of the project and the constraints 
- If the project involves the solution of a specific problem or the production of a specific system this should be clearly specified in an informal way.
- ginally, the introduction should summarise the remaining chapters of the report, in effect giving the reader an overview of what is to come.

\section{Chapter 2: Literature Survey}

Depending on the type of the project, relevant literature may be hard to find, and a technology survey/review of relevant mathematics/review of similar software tools may be more appropriate - you should discuss this with your supervisor. 
A good literature survey should demonstrate your awareness and understanding of the background literature to your topic. 
It should begin by setting the proposed research in a wide context, and progress to a more detailed account of the most relevant work in the area, taking care to include some up-to-date references. Reviewing the literature can help to identify questions and issues that have not yet been answered, ideally questions that will be addressed through your project. It may also be appropriate to incorporate criticisms of previous work, although you need to take care here that your criticisms do not reflect a lack of understanding. 

Think of the review as writing an essay on the background literature for your project. 
You should not just provide a list of references followed by a short summary of each of them. Instead the review should be organised and structured in a meaningful way, and the themes and relationships between the references identified. 
You should expect to redraft the review several times in order to arrive at a text that is clearly written, easy to understand, but that displays an in-depth understanding of the topic. 

\section{Chapter 3: Requirements and analysis}

Detail the aims and objectives of your project and analyse individual parts in detail. 
The analysis may cover more than is finally implemented. As a result of the analysis, you should state what will be covered by the project and what will not be done and why. 
Due consideration should also be given to how you will evaluate your work. 
Evaluation is one of the most important aspects of any piece of work and it should be thought about in the early stages. 
Consider tests or experiments that can be conducted to establish the success of the work.


\section{Chapter 4: Conclusions, progress to date and project plan}


\subsection{achievements to date}

\subsection{Gantt Chart until the end}

\section{Appendices}
- class interfaces
- database schemas
- log of milestones

\section{Citations}

\newpage

\end{document}
