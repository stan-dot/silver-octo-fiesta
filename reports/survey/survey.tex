\documentclass{article}
\usepackage{forloop}
\newcounter{loopcntr}
\newcommand{\rpt}[2][1]{%
  \forloop{loopcntr}{0}{\value{loopcntr}<#1}{#2}%
}
\newcommand{\on}[1][1]{
  \forloop{loopcntr}{0}{\value{loopcntr}<#1}{&\cellcolor{gray}}
}
\newcommand{\off}[1][1]{
  \forloop{loopcntr}{0}{\value{loopcntr}<#1}{&}
}

\title{Dissertation Project Description}
\author{Stanislaw Malinowski}
\subitem{Rob Gaizauskas}
\subitem {COM3610}
\date{October 2022}

\begin{document}

\maketitle
\section{title}
Title, name, supervisor, module code, date,
and the following statement: This report is submitted in partial fulfilment of the requirement for the degree of [Degree Title] by [Full Name].

\section*{Contents}
\tableofcontents
\newpage
\section{Dissertation Project: Survey and Analysis Stage}

\section{Declaration}
All sentences or passages quoted in this report from other people's work have been specifically acknowledged by clear cross-referencing to author, work and page(s). Any illustrations that are not the work of the author of this report have been used with the explicit permission of the originator and are specifically acknowledged. I understand that failure to do this amounts to plagiarism and will be considered grounds for failure in this project and the degree examination as a whole
\newpage

\section{Abstract}
- background
- project aims
- achievements to date

\section{Chapter 1: Introduction}
How to get beyond bag-of-words approaches for meaning and sentiment analysis? 
How to get to the structure of the argument, that may escape even the utterer?
These are questions that interested researches in discourse analysis for many years.

Data science has been applied with incredible success in many NLP tasks, sentiment analysis market valued at \$3.15 billion in 2021 (https://www.polarismarketresearch.com/industry-analysis/sentiment-analytics-market)
The results are limited. Opinions gathered in this way are devoid of reasoning structure.
The problem in many studies of argumentation is the undersupply of data.
Yet there is plenty of online discourse.
Online platforms, such as reddit, quora, twitter are host to the most resounding debates, heard by hundreds of millions. 
Discussions topics inlcude current political controversies, religious questions and [lore discussions].
These discussions form the core of the public discourse in modern society. 
The act of commenting online corresponds to 'claiming ownership for a new piece of knowledge' (Teufel et al, 2009).
Expressing arguments for, negotiating compromises, pursuing social shifts, finding or failing-to-find common ground - are all different types of discourses. 
The usual output is a series of posts, comments or tweets, that are understructured. 
Previous studies included attempts to automativally derive representations of the discourse structure from unstructrued text (Anand et al 2011).
The main drawback is the annotation cost, using dozens of hours of expert work, and tens of pages of annotation manuals.

This project is focused on data gathering. The goal is to investigae the 'games with purpose' paradigm to make crowdsourcing viable in this area.
This report will go over the relevant literature, inlcuding human annotation schemas, and semi-supervised learning approaches, then review similar software. Based on that an outline of requirements is given. 
That section will be the basis for development of a more technical design document on a later date.
Report will end with a description of achievements to date and plans for the future.

\section{Chapter 2: Literature Survey}
Argument structure analysis goes back to antiquity. I. Angelelli. The techniques of disputation in the history of logic, 1970.
Analysis of argumentation has been an active topic in numerous research areas, such as philosophy (van Eemeren et al., 2014), communication studies (Mercier and Sperber, 2011), and informal logic (Blair, 2004), among others
This chapter will cover the areas that this project touches.

There are types of approaches. Fully annotated and  semi-supervised approaches.

\subsection{annotations - static argumentation schemes}
There are different strategies for annotation. Much of the study of annotation has focused on citation function in scientific papers.
Many annotation schemas for argumentcitation motivation have been created (Teufel et al 2010, Mann and Thompson 1987) and had success in research.
These studies work by outlining analytically from a sample of essays, articles or other texts, an exhaustive list of types of roles of statements.
Then the researchers proceed to create a long, multi-page annotation guidelines. Up to 111 sides of A4 in some cases! (Teufel et al 2009).
As the next step, annotators are selected. Some studies choose discipline experts, some on purpose select persons not familiar with the discipline.
That is done in order to increase domain-independence of the annotation schema.
The following step is usually to collate the annotations created by the annotators. 

Study by Siddarthan and Tidhar from 2006 'automatic classicification of citation function' describes use of Kappa coefficeint for that purpose.
The paper describes the use of Kappa coefficeint thus:
> coefficient κ (Fleiss, 1971; Siegel and Castellan, 1988), the agreement measure predominantly used in natural language processing research (Carletta, 1996). κ corrects raw agreement P(A) for agreement by chance P(E)

In summary, the pure annotation strategy is costly but its experiences are quite helpful. The basic information flow pipeline is shown in (Razuvayevskaya and Teufel 2017 - finding enthymemes in real-world text - a feasibility study).
> mining: argument extraction, segmentation, i.e. identification of minimal argumentative discourse units (ADUs), segment classification, i.e. labeling of ADUs based on their argumentative roles, identification of relations between these segments, and argument completion, i.e. automatic construction of statements from implicit propositions
The latter part includes enthymeme decection, which is defined as:
> According to the Aristotelian definition [6], enthymemes are standard-form syllogisms with one missing proposition.  (ibid.)

For the crowdsourced model the 5 steps can be traced for each of the parts of the test corpus. 

\subsection{annotation with crowdsourcing}
A number of researchers explored the crowdsourcing area of the potential solution space to the problem of argument structure extraction.
Following (von Ahn, 2008), we can distinguish between 3 types of approches to crowdsourcing with respect to the type of incentive provided to the participants.
There is the financial approach, the altruistic approach, and the gamification (enjoyment) approach. Papers reviewed did not feature the enjoyment approach.

Starting chronologically, the first paper is Anand et al 2011 "How can you say such things?!?: Recognizing Disagreement in Informal Political Argument".
One of the features of this paper to note is its use of Amazon's "Mechanical Turk" service for contributions. Paid experts did the annotations.
Another one is the choice of possible labels. Thse are 'agree/disagree', 'fact/emotion/, 'attack/insult', 'sarcasm', 'nice/nasty'

Another example comes from (Wyner, Peters, and Price 2015 - arugment discovery and extraction with the Argument Workbench).
Quotes are a description of the tool
> ArgumentWorkbench, which is a interactive, integrated, modular tool set to extract, reconstruct, and visualise arguments. [...]
> The Argument Workbench is a processing cascade, developed in collaboration with DebateGraph. [...]
> it is an open source desktop application written in Java that provides a user interface for professional linguists and text engineers to bring together a wide variety of natural language processing tools and apply them to a set of documents
It is worth noticing that they used a desktop application. The other option being mobile application, there are tradeoffs visible.
Mobile applications don't have the 'screen real estate' for complex interfaces, yet higher number of people use them. Over 50\% more people use mobile.
https://gs.statcounter.com/platform-market-share/desktop-mobile-tablet
For low-entry-cost crowdsourcing approach that is essential.

Then the authors mention the workflow, and attach a diagram.
> We harvest and preprocess comments; highlight argument indicators, speech act and epistemic terminology; model topics; and identify domain terminology.
> we use the GATE framework (Cunningham et al.(2002)) for the production of semantic metadata in the form of annotations

\subsection{online current corpora}
There are some ready corpora for argumnet minig.
This section will focus on their formats, topics and limitatinos.

Studies mention source data themselves. 
For instance, (Awadallah, Ramanath, and Weikum 2012 Harmony and dissonance: organizing the people's voices on political controversies)
say the following:
> For matching names in text sources against canonical entities, we leverage existing knowledge bases like DBpedia, Freebase, or Yago.

Moreover, Argument mining survey () mentions many of them.
https://direct.mit.edu/coli/article/45/4/765/93362/Argument-Mining-A-Survey
> Internet Argument Corpus (IAC) (Walker et al. 2012) is a corpus for research in political debate on Internet forums. It consists of approximately 11,000 discussions, 390,000 posts, and some 73,000,000 words
> AIFdb17 (Lawrence et al. 2012), containing over 14,000 Argument Interchange Format (AIF) argument maps, with over 1.6m words and 160,000 claims in 14 different languages.18 These numbers are growing rapidly, thanks to both the increase in analysis tools interacting directly with AIFdb and the ability to import analyses produced with the Rationale and Carneades tools (Bex et al. 2012). Indeed, AIFdb aims to provide researchers with a facility to store large quantities of argument data in a uniform way. AIFdb Web services allow data to be imported and exported in a range of formats to encourage re-use and collaboration between researchers independent of the specific tools and data format that they require.
https://corpora.aifdb.org/
Both corpora focus on manual labeling.
The Argument Interchange Format seems as a valid standard for any argument crowdsourcing project to follow.
This standardized JSON graph format make data transport and display easier. That is a great opportunity as their platform features a display tool. 
The project can leverage this external tool and standard and focus on other features of the crowdsourcing process.
Another resource of arguments online is the `args.me` online resource. 
Its creation was described in Wachsmuth, Stein, and Ajjour 2017 - buildign an argument search engine for the WEB.
Args.me has an exposed search API and database schemas. The paper also emphasises the ethical choice contained in the construction of personalized search.
It also mentions the high cost of the long hours of expert time.
Another resource is the Araucaria program, mentioned by several papers. Link provided seems to be broken, though. http://araucaria.computing.dundee.ac.uk/

There are cons to the current corpora, though.
Only few publicly available argumentation corpora exist, as annotations are costly, error-prone, and require skilled human annotators (Stab and Gurevych, 2014a; Habernal et al., 2014).
Although argumentation mining in user-generated Web discourse has a long way to go (our methods currently achieve only about 50\% of human performance)

\subsection{mixed approaches}
The above approaches were deemed to be limited. In fact, there seems to have been a shift in approaches around 2014.
There has been an argument that automatic processing is preffered due to the cost, unreliablity of annotations and scarcity of trained personnel
(Stab and Gurevych, 2014a; Habernal et al., 2014)
With that being the turining point, the efforts thereafter have focused on putting the main weight of the process on automatized process. 
That reflects the wider industry shift towards the use of Machine Learning over the past 10 years.

Researchers conceded that manual analysis is not feasuble some studies (Habernal and Gurevych 2015 - exploiting debate portals for semi-supervised).
They attempted to bypass that problem using semi-supervised learning.

Another approach is a blending approach. That consists in adding small amount of high quality and manually labeled data.
That vein of thoought is explored in (Will it Blend? Blending Weak and Strong Labeled Data in a Neural Network for Argumentation Mining Eyal Shnarch et al.
ACL, 2018 http://aclweb.org/anthology/P18-2095)
Existence of this approach fares well for the crowdsourcing approach. Primarily the data will be strongly labeled, but if it proves weak for training using ML techniques, 
supplementation with weakly labeled data will be saving the utility of crowdsourced data.

Another attempt was this paper (Al-Khatib et al 2016 - crossdomain mining of argumentative text through Distant Supervision).
This study makes the strongest case for the exclusion of crowdsourcing as a means of data acquisition.
> Studies reveal that annotators need multiple training sessions to identify and classify argumentative segments with moderate inter-annotator agreement, and crowdsourcing-based annotation does not help notably (Habernal et al., 2014)
Of note is also the source of the data. Authors say
> we acquire a large corpus with 28,689 argumentative text segments from the online debate portal idebate.org. The corpus covers 14 separate domains with strongly varying feature distributions.
This source is highly specifc. That argumentative discourse is not a representative sample of human argument space. Verbal arguments
also experts 

These arguments are put into the search system using the PageRank algorithm (Brin et al 1998). 
That gives grounds to considering that algorithm a good basis for a ranking of arugments in other circumstances.

\subsection{Games with purpose to the rescue}
How to apply GWP approach?
A guiding paper to this area of literature review was 2008 paper by von Ahn et al, Designing games with a purpose https://dl.acm.org/doi/10.1145/1378704.1378719
Main takeaways from this paper is what makes games successful.
These are three main factors: enjoyment, timed response, ELO system.
"Enjoyment" is not defined formally there, but simply a result of knowing the games by their fruits:
> The key property of games is that people want to play them. We therefore sidestep any philosophical discussions about “fun” and “enjoyable,” defining a game as “successful” if enough human-hours are spent playing it
That  *a posteriori* feature can be predicted by keeping the right level of challenge in front of the player. (Locke, E.A. and Latham, G.P. A Theory of Goal Setting and Task Performance. Prentice Hall, Englewood Cliffs, NJ, 1990.)
Right level of challenge is measured through a 'timed response' mechanic built into the game.
The player's skill is evaulated on the basis of duration of the task. 
Self awareness of the speed of execution and the increasingly narrowing bar representing time left provides live feedback.
ELO system ensures that in player versus player games there is balance. Each player can't go against player very far from their own skill level.
Moreover, it exploits the competitive desire in humans.
The social factor seems valued in games as competitive sports are a growing market, with nearly 2 million US dollars revenue predicted in 2025
https://www.statista.com/statistics/490522/global-esports-market-revenue/
Social factors have not been observed to feature in the studies mentioned so far.
There was no competitiveness, the workflow was each annotator working individually.
There was no incentive for agreement between annotators.
This is a limnitation that could be overcome.

The other takewaway is a set of metrics. These are
- labels per human hour
- Average Lifetime Play
While the first one is self-explanatory, the second might not be.
Study describes the latter as 
> ALP is the overall amount of time the game is played by each player averaged across all people who have played it. For instance, on average, each player of the ESP Game plays for a total of 91 minutes.
Average score of 91 minutes is a good benchmark to compare with.

https://dl.acm.org/doi/10.1145/2448116.2448119
Phrase detectives as the leading example of the use of the 'game with purpose' paradigm in order to gather data for NLP research.
ALP of only 30 minutes. It seems there is a a headspeace to improve on that. nothing indicates that this area has hit a ceiling.
What they excel at is: 'task completion, scoring and storyline as a seamless experience'
These are qualities worth emulating. Task completion and scoring are the easiest to implement. 
In contrast, creating a coherent and engaging storyline would require a specific skillset that might not always be available.
Possibly emphasizing social features could substitute this.

\section{Chapter 4: Similar software review}
There is a plethora of similar software. 
There are commrecial B2B SAAS solutions, aiming to help in meetings, by providing a way to write structured notes and diagrams.
There are also hacker-ethics personal knowledge management tools.
There is also a record of software developed specifically for research.

\subsection{Business solutions}
There are many companies offering business-to-business services in NLP. 
These range from meeting minutes or report summaries, live structured note taking, visualization software, or customer sentiment analysis.
Business model is frequently 'freemium', where users can use the app for free, but some features are hidden behind a paywall.

I will only give a brief summary of each:

Lexikat provides "no-code concept maps and text analysis models from any document."
https://lexikat.com/#/

Infranodus is a multi-purpose analytics tool, extracting arguments and enhancing it through various AI techniques. 
It is unknown whether the analysis uses argument mining.
https://infranodus.com/

Lexisnexis provides analytic tools in the legal domain, also providing visualization
https://www.lexisnexis.com/en-us/home.page

Summetix uses Argument Mining to provide summaries of trends in customer behaviour.
https://www.summetix.com/


Reasoning lab is the creator of the Rationaleonline service described below
https://www.reasoninglab.com/argument-mapping/

Rationaleonline - a huge collection of public maps. Of limited use as a dataset for studies in English, as most of the data is not in English.
https://www.rationaleonline.com/browse/all

Mindup - a tool for creating 'mind maps'.
https://www.mindmup.com/

Crowdee is a service providing crowdsourced contributions to business customers, with the specific niche of creating ML training datasets.
https://www.crowdee.com/

Amazon Mechanical Turk is a service acting as a middleman between business seeking temporary increase in workforce to solve a specific issue.
The service accomplishes this through splitting the workload into 'microtasks' that are then handled by a destributed workforce.
https://www.mturk.com/


\subsection{Personal knowledge management tools}
Argunet (started in 2006) and the successsor project argdown is a tool for argument representation.
What is exceptional about the Argdown is that it provides a textual standard to represent arguments of any complexity.
http://www.argunet.org/
https://argdown.org/

Planner is the pioneer in the field, even if not exactly with *personal* applications in mind.
It was created in 1969 for forward and backward chaining of statements to prove some goal from the database.
https://en.wikipedia.org/wiki/Planner_(programming_language)
The function of it was thus *synthetic*, not analytic - so the reverse of argument mining. 
Nevertheless, it is an interesting reference point.

\subsection{Research-grade software}

Started in 2000 and with the latest release in 2020, Flora2 is a system for knowledge representation using XSB for inference
As it is clear from the website look and feel, this software is not for mass adoption.
With less thatn 600 downloads over the last year it does not seem to have taken off.
https://sourceforge.net/projects/flora/files/stats/timeline?dates=2021-12-04\%20to\%202022-12-03&period=daily

https://flora.sourceforge.net/

Zooniverse is a thriving  crowsourcing platform. 
There are many projects in the categories such as medicine, nature, climate and history.
Not many projects in language processing domain have been found yet.
One example is an Optical Character Recognition crowdscouring dealing with 19th cenutryo newspapers.
https://www.zooniverse.org/projects/bldigital/living-with-machines
https://www.zooniverse.org/projects

Insofar as the argument mining relates to Knowledge Graphs and Ontology extraction, Kelpie is one of the frameworks for the former.

Authors say Kelpie is a:
> XAI framework for interpreting Link Predictions on Knowledge Graphs
https://github.com/AndRossi/Kelpie
It is in active development and it might be useful for downstream processing of the data obtained through crowdsourcing.

\section{Chapter 3: Requirements and analysis}

\subsection{Requirements}
To avoid the pitfalls of human annotation, a quantitative approach should be pursued. 
A diverse range of both source texts and human participants is needed.
Through the 'games with purpose' paradigm, we can expect users/ players to contribute for their own enjoyment (von Ahn, 2014).
Enjoyment maximization should result in a higher number of Average Lifetime Contribution(ibid.).
A spotless user flow and an intuivie application are of the highest priority to achieve that.

A sample user flow can be imaged as the following:
- Player finds the landing page
- Browses the list of topics popular at the moment
- clicks to read more, is redirected to the app page
- there is presented with a tutorial how to engage with the systemn: read, and contirbute.

Over the following week:
- user completes a small number of interactions with the application
- user intergrates the app importing some content on their own, be it from Twitter or other dynamic corpora (for instance a blog article to annotate to understand it better)
- user continues to grow their use of the app, perhaps using it as a note taking app


\subsection{Design decisions to be taken}
There is the option for a debate mode for the app.
That would be a type of game, where addition of a node to the argument tree would constitute a move.
That type of games was examined in Prakken 2005.
There were distinguished different types of 'reply protocols' - unique vs multi-reply, immediate and non-immediate, uniqe, multi-move.
The configuration space is non-trivial, and the choice which of these protocols would be implemented in the game needs to be pondered.

(Prakken 2005, Coherence and Flexibility in Dialogue Games for Argumentation)

Another distinction between game-states comes from Rienks, Heylen and Weijden 2005 - argument diagramming of meeting conversations.
That is the observations that there are more types of issues than just 'yes/no' issues.
They call them: A/B isses, yes-no issue, open issue.
It is obvious that the yes-no issue might be the simplest to implement, yet it would be a lost opportunity to forsake the other modes.
Some consideration should be given to incorporating them into the space of possible arguments on the app.

As mentioned above, AIFdb format may be used for data portability, and Argdown for markdown notation.

\subsection{success criteria}
The final criteria is the volume of quality annotated corpus. It could be uploaded to AIFdb. 
This will be known after performing quality control on the main argument database created through the lifetime of the app. 

The other criterion is the number of downloads - if larger than 150 better than global average, if more thatn 1300, over global average.
https://www.statista.com/statistics/1119893/average-number-downloads-united-states-app-publishers/

These data will be achieved from app publisher analytics.

\section{Chapter 4: Conclusions, progress to date and project plan}
\subsection{Achievements to date}
To date the main achievement is exploration of the possibilities of mobile app development. 
Expo library was chosen for the simplicty of deployment to host the app https://expo.dev/, while a Nextjs website will serve as a landing page.
There was some work - a demo of a drag-and-drop implemented in Reactjs. The full interactive interfce proved harder than expected, and in the light of further discoveries of extant work, has proven superfluous.
Therefore it was not finished.
The OVA platform laready performs graph visualization quite well.
http://ova.arg-tech.org/analyse.php?url=local&aifdb=1238&akey=dd8dd1dfd567f735f46df14ff08a0fc5

\subsection{Gantt Chart until the end}
% todo add gantt chart from here https://www.mjr19.org.uk/IT/gantt_latex.html
% items of the gantt chart
% bare bones connected structure
% SEO optimized website
% design document
% app MVP
% gathering data
% adding new tasks 
% adding new source datasets
% promotion
\definecolor{orange}{RGB}{255,127,0}

\noindent\begin{tabular}{p{0.17\textwidth}*{20}{|p{0.01\textwidth}}|}

\noindent\begin{tabular}{p{0.17\textwidth}
!{\vrule width 0.4mm}p{0.01\textwidth}*{3}{|p{0.01\textwidth}}
!{\vrule width 0.4mm}p{0.01\textwidth}*{3}{|p{0.01\textwidth}}
!{\vrule width 0.4mm}p{0.01\textwidth}*{3}{|p{0.01\textwidth}}
!{\vrule width 0.4mm}p{0.01\textwidth}*{3}{|p{0.01\textwidth}}
!{\vrule width 0.4mm}p{0.01\textwidth}*{3}{|p{0.01\textwidth}}
|}
% The top line
\textbf{Gantt chart} & \multicolumn{4}{c!{\vrule width 0.4mm}}{Week 1} 
           & \multicolumn{4}{c!{\vrule width 0.4mm}}{Week 2} 
           & \multicolumn{4}{c!{\vrule width 0.4mm}}{Week 3} 
           & \multicolumn{4}{c!{\vrule width 0.4mm}}{Week 4} 
           & \multicolumn{4}{c|}{Week 5} \\
PhD 1     \on[16] \off[4]\\
\hline
Squashing Bugs \off[2] \on[4] \off[4] \on[1] \off[9] \\
\hline
\textbf{Work \mbox{packages}} & \multicolumn{20}{c|}{} \\


\section{Citations}
\bibliographystyle{}
\bibliography{}
Teufel, Siddharthan and Batchelor  2009 - towards discipline - independent argumentative zoning - evidence from chemistry and computationl linguistics
\newpage
\end{document}
