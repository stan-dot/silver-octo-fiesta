

The theatre of these exchanges is the online platform.
That increases the importance for automatic processing of that data.
Monologues are represented better on twitter, with threads being a linear structure. Branching structures are less legible.
That can hostilities.

Exposure to alien views is much more common in online communities compared to offline ones (Anand et al, 2012).
On the other hand, homophily is also present (Cinelli et al, 2021) https://www.pnas.org/doi/10.1073/pnas.2023301118
In any topic, a relatively small number of arguments is repeated multiple times in 'echo chambers'.
Many users base their stance on those most common arguments. These are often expressed in different wording.
Linking textual statements to these abstract arguments is non-trivial. (Boltuzic, Snajder 2015)

There are more appplications to this automatization.
It was remarked ( Carstens and Toni 2015) that there are many applications of the automatic processing of argumentative data,
 such as automated decision making (Bench-Capon et al., 2009) or pro-and-con search engines (Cabrio and Villata, 2012c).



\subsection{App use sample screen}
% todo this weekend https://wireframe.cc/

\subsection{database schemas}
% todo this weekend

