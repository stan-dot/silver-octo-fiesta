\documentclass{article}
\usepackage{blindtext}

\title{Sections and Chapters}
\author{Stanislaw Malinowski}
\subitem{Supervisor}
\subitem {module code}
\date{\today}

\begin{document}
\maketitle
\section{Title page}
Title, name, supervisor, module code, date, and the following statement:    

"This report is submitted in partial fulfilment of the requirement for the degree of [Degree Title] by [Full Name]".   

The title the dissertation ends up with need not be the one it started with in the project choice stage more than a year earlier but it should be meaningful.  "My Design Project" is not meaningful.
\blindtext
\newpage

\section{Declaration}
All sentences or passages quoted in this report from other people's work have been specifically acknowledged by clear cross-referencing to author, work and page(s). Any illustrations that are not the work of the author of this report have been used with the explicit permission of the originator and are specifically acknowledged. I understand that failure to do this amounts to plagiarism and will be considered grounds for failure in this project and the degree examination as a whole.

(your name**)"
\blindtext
\newpage

\section{Abstract}
This should be two or three short paragraphs (100-150 words total), summarising the dissertation. It is important that this is not just a restatement of the original project outline. A suggested flow is background, project aims and main achievements. A bad abstract would have a final paragraph that just said "the achievements will be described" - this is useless, as it says nothing. From the abstract a reader should be able to ascertain if the project is of interest to them and presents results of which they would like to know more details.
\blindtext
\newpage

\section{Acknowledgements}
Thanks to whoever may have helped you in any way - both serious and a bit of fun.
\blindtext

\newpage

\section{Contents}
\tableofcontents
\blindtext
\newpage

\section{Chapter 1: Introduction}
\blindtext
\newpage

\section{Chapter 2: Literature review}

\subsection{Topic area 1}
\blindtext
\subsection{Topic area 2}
\blindtext
\subsection{Topic area 3}
\blindtext
\newpage

\section{Chapter 3: Requirements and analysis}
\begin{itemize}
  \item 
\end{itemize}
\blindtext
\newpage

\section{Chapter 4: Design}
https://adancewithbooks.wordpress.com/2019/09/22/a-small-list-of-trigger-warnings-you-can-use/

\subsection{User Resources}
\begin{itemize}
  \item experience points - 
  \item progression - new capabilities unlocked  
  \item competition - leaderboard feature, in different game modes
  \item achievements - badges for completing certain milestones, for each game mode
  \item altruism
\end{itemize}

\subsection{Game Modes}
\begin{itemize}
  \item Main mode - varied tasks and rewards
  \item Complex mode
  \item confirmation mode - grinding some kind of points
  \item daily challenge mode
  \item seasonal modes
\end{itemize}

\subsection{User journey}
\begin{itemize}
  \item completes the tutorial
  \item logs to the website
  \item 
\end{itemize}
\blindtext
\newpage

\section{Chapter 5: Implementation and testing}
\blindtext
\newpage

\section{Chapter 6: Results and discussion}

\blindtext

\subsection{Further directions}

\newpage

\section{Chapter 7: Conclusions}
\blindtext
\newpage

\section{References}
\blindtext
\newpage

\section{Appendices}
\blindtext
\newpage

\section{Ethics statement}
\blindtext
\newpage
\end{document}